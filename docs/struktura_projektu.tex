%!TEX root = ./report.tex

\section{Struktura projektu}
	\label{final:struktura}

	\subsection{Wymagania systemowe}
		\label{final:struktura:wymagania}

		Do testowania został zastosowany python w wersji 3.6.3 na systemie operacyjnym Ubuntu. Obliczenia wykonywane były na komputerze z wielordzeniowym procesorem co pozwoliło na zrównoleglenie obliczeń, co znacznie zmniejszyło czas wykonywanych obliczeń. Ważniejsze zastosowane biblioteki python'a:
		\begin{itemize}
			\item argparse - biblioteka zawierająca funkcjonalności pozwalająca na wygodne parsowanie argumentów wejściowych aplikacji
			\item multiprocessing - biblioteka pozwalająca na wielowątkowe obliczenia
			\item logging - wyświetlanie logów w konsoli
			\item json - konwersja danych do formatu json
			\item statistics - biblioteka zawierająca funkcje pozwalające na obliczenie średniej i odchylenia standardowego
			\item gzip - pakowanie generowanych grafów do formatu .gz, w celu oszczędności pamięci na dysku
			\item matplotlib - biblioteka pozwalająca na rysowanie wykresów
		\end{itemize}

	\subsection{Podział na moduły}
		\label{final:struktura:klasy}

		% Nie trzeba opisywać dokładnie modułów, po to jest dokumentacja w kodzie. Fajnie by było jednak wstawić tutaj jakiegoś UMLa, który opisuje zależność między modułami/klasami (bo tego w dokumentacji w kodzie nie da się umieścić).

		% Napisać, że o ile struktura grafu i węzła została już opisana wcześniej (dać ref'a do zmian i projektu), to kilka słów wyjaśnienia należy się klasom, których jeszcze nie opisywaliśmy. I dosłownie, napisać po trzy zdania dla każdego, co robią.
		Aktualna struktura grafu \ref{final:zmiany} została opisane wcześniej, lecz konieczne jest również krótkie wyjaśnienie wcześniej nieopisywanym klasom.
		\subsubsection{Klasa \texttt{GraphGenerator}}
			Klasa implementująca mechanizm generacji grafu. Jako argumenty przyjmuje liczbę wierzchołków grafu oraz promień. W pierwszej kolejności generowane są współrzędne wierzchołków. Następnie tworzona jest lista krawędzi grafu. W tym celu analizowane są wszystkie pary wierzchołków i jeśli odległość między nimi jest mniejsza od promienia to jest tworzona krawędź.
		\subsubsection{Klasa \texttt{GraphDatabase}}
			Klasa zawierająca mechanizmy zapisu oraz odczytu grafów z pliku celem oszczędności czasu obliczeń oraz celem późniejszego wykorzystania do wielokrotnych testów.
		\subsubsection{Klasa \texttt{GraphAnalyzer}}
			Implementuje analizę grafu. Oblicza maksymalny rozmiar składowej spójnej, prawdopodobieństwo spójności grafu(średnią oraz odchylenie standardowe). Wykonywane obliczenia są zapisywane do pliku w formacie json.
		\subsubsection{Klasa \texttt{ResultsPlotter}}
			Klasa pozwalająca na generowanie wykresów, które umieszczone zostały w niniejszej dokumentacji.

	\subsection{Sposób uruchamiania}
		\label{final:struktura:uruchamianie}

		\noindent W celu uruchomienia programu należy wywołać plik \texttt{analyze\_graphs.py}. Przyjmuje on następujące parametry:
		\begin{itemize}
			\item \texttt{---start\_size} - wymagany. Liczba wierzchołków pierwszego generowanego grafu

			\item \texttt{---stop\_size} - wymagany. Liczba wierzchołków ostatniego generowanego grafu

			\item \texttt{---size\_step} - opcjonalny. Liczba wierzchołków dodawana do kolejnego generowanego grafu.

			\item \texttt{---start\_radius} - wymagany. Promień pierwszego generowanego grafu

			\item \texttt{---stop\_radius} - wymagany. Promień ostatniego generowanego grafu

			\item \texttt{---radius\_step} - opcjonalny. Promień dodawany do kolejnego generowanego grafu

			\item \texttt{---repeats} - opcjonalny. Liczba testów(powtórzeń) wykonywanych w ramach jednego grafu

			\item \texttt{---jobs} - opcjonalny. Liczba wątków, przy użyciu których będą wykonywane obliczenia

			\item \texttt{---output\_dir} - opcjonalny. Folder, do którego będą zapisywane wyniki

			\item \texttt{---verbose} - opcjonalny. Spowoduje, iż na konsolę będą wypisywane komunikaty diagnostyczne (pomocne, gdy obliczenia trwają bardzo długo)
		\end{itemize}

		Przykładowy sposób wywołania:

		% TODO: sformatować to, żeby nie było jednym szlaczkiem
		\texttt{python3 ./analyze\_graphs.py -v ---start\_size=10000 ---stop\_size=20000 \\---size\_step=100 ---start\_radius=0.01 ---stop\_radius=0.02 \\---radius\_step=0.001 ---jobs=8 ---repeats=100 ---output\_dir=output}


		Takie wywołanie spowoduje, że zostaną wowołane testy dla grafow o rozmiarze od 10000 do 20000 wierzchołków z krokiem 100 oraz promieniem w zakresie od 0.01 do 0.02 z krokiem 0.001. Dla każdego przypadku zostanie wygenerowane 100 grafów. Obliczenia będą wykonywane z użyciem 8 wątków. Wyniki zostaną zapisane w folderze output.
