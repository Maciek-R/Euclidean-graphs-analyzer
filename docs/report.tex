%!TEX encoding = UTF-8~Unicode
%!TEX program = pdflatex
%!BIB program = biber

\documentclass[11pt,a4paper,twoside]{report}
\usepackage[a4paper, top=25mm, bottom=25mm, inner=30mm, outer=30mm, twoside]{geometry}

\usepackage[polish]{babel}
\usepackage[T1]{fontenc}
\usepackage[utf8]{inputenc}
\usepackage{lmodern}
\usepackage{amsmath}
\usepackage{url}
\usepackage{fancyhdr}
\usepackage{footnote}
\usepackage{float}

% Kody źródłowe
\usepackage{listings}

% Algorytmy (pseudokod)
\usepackage[algochapter,boxed,lined,linesnumbered]{algorithm2e}
\renewcommand{\algorithmcfname}{Algorytm}

% Kolory kodu
\usepackage{xcolor}

% Linki w~bibliografii
\usepackage{url}

% Przecinki zamiast kropek w~liczbach
\usepackage{icomma}

% Generalna typografia
\usepackage{microtype}

\usepackage[unicode,pdfusetitle,bookmarksopen,bookmarksnumbered]{hyperref}
\usepackage{graphicx}
\usepackage{array}
\usepackage[font=footnotesize,format=hang,labelfont=bf]{caption}

% Wcięcie pierwszego akapitu każdej sekcji
\usepackage{indentfirst}
\newcommand{\HRule}[1]{\rule{\linewidth}{#1}}

\fancyhf{}
\renewcommand{\headrulewidth}{0pt}
\pagestyle{fancy}
\fancyfoot[LE,RO]{\thepage}

%%% Bibliografia
% Zamiana cudzysłowów (dla biblatexa)
\usepackage{csquotes}

% Lepsza bibliografia
\usepackage[backend=biber,style=numeric,sorting=nty,abbreviate=false,language=polish]{biblatex}
\addbibresource{report.bib}
\renewcommand*{\newunitpunct}{\addcomma\space}
\DeclareQuoteAlias{croatian}{polish}

\clubpenalty=10000
\widowpenalty=10000


\begin{document}
%-------------------------------------------------------------------------------
\title{ \normalsize \textsc{Grafy i~Sieci\\Dokumentacja projektowa} \\ [2.0cm]
		\LARGE \textbf{Badanie właściwości grafów euklidesowych} \\ [2.0cm]
		\normalsize \today \vspace*{3\baselineskip}}
\date{}
\author{Adam Kowalewski\\Maciej Ruszczyk}
\maketitle
\tableofcontents
\newpage
%-------------------------------------------------------------------------------

%!TEX root = ./report.tex

\chapter{Wprowadzenie}

Celem zadania projektowego jest zaimplementowanie generatora sieci euklidesowych, a~następnie zbadanie prawdopodobieństwa spójności sieci i~określenie rozmiaru największej składowej spójnej w~zależności od liczby i~zasięgu wierzchołków. Zarówno generator, jak i~oprogramowanie weryfikujące zostaną napisane w~języku Python. Głównym źródłem wiedzy będzie książka autorstwa Jacka Wojciechowskiego i~Krzysztofa Pieńkosza ,,Grafy i~Sieci''.\cite{WojciechowskiJacek2013Gis}

%!TEX root = ./report.tex

\chapter{Projekt rozwiązania}
	\label{projekt}

	\section{Struktury danych}
	\label{projekt:struktury}

	%TODO: Dać rysunek poglądowy struktur danych

	\noindent W projekcie zostaną wykorzystane następujące struktury danych:
	\begin{itemize}
		\item Struktura opisująca graf - \texttt{G}. Składać będzie się ona z:
			\begin{itemize}
				\item Tablicy wierzchołków \texttt{V} o~rozmiarze \texttt{n},
				\end{itemize}

		\item Struktura opisująca wierzchołek - \texttt{v}. Składać będzie się ona z:
			\begin{itemize}
				\item Współrzędnej $\texttt{x} \in [0; 1]$
				\item Współrzędnej $\texttt{y} \in [0; 1]$
				\item Listy z~numerami sąsiadujących wierzchołków \texttt{U}
			\end{itemize}
	\end{itemize}

	\section{Opis algorytmów}
	\label{projekt:algorytmy}

	Programy zaimplementowane w~ramach projektu wykonywać będą zasadniczo dwie czynności: generować sieć euklidesową oraz badać parametry spójności. Poniżej opisano zaproponowane algorytmy.

	\subsection{Generacja sieci euklidesowych}
		\label{projekt:algorytmy:generacja}

		%TODO: Dać listing algorytmu generacji
		%TODO: Dać przykładową sieć

		Algorytm ten będzie przyjmował dwa dodatnie parametry wejściowe: liczbę wierzchołków \texttt{n} oraz promień \texttt{r}. Na początku algorytm generuje odpowiednią ilość wierzchołków, w~sposób losowy, na obszarze 2D o~współrzędnych $(\texttt{x}, \texttt{y})$ mieszczących się w~zakresie $[0; 1]$. Następnie, dla każdej pary wierzchołków $(\texttt{v}, \texttt{u})$, obliczana jest odległość między nimi, w~oparciu o~metrykę euklidesową. Jeśli odległość ta jest mniejsza lub równa promieniowi, między wierzchołkami tworzona jest krawędź (dodawany jest nowy sąsiad).

	\subsection{Badanie spójności sieci}
		\label{projekt:algorytmy:spojnosc}

		Algorytm ten zostanie zaimplementowany w~oparciu o~technikę badania grafu w głąb\cite[s.~270]{WojciechowskiJacek2013Gis}. Na wejściu przyjmować będzie sieć wygenerowaną w~punkcie~\ref{projekt:algorytmy:generacja}. Na wyjściu zaś zwrócony zostanie rozmiar największej składowej spójnej. Jeśli dany graf jest spójny, to otrzymana wartość będzie równa ilości krawędzi. Pseudokod opisujący algorytm przedstawiono na listingu~\ref{alg:spojnosc}.

		\begin{algorithm}
			\SetKwData{maxSize}{maxSize}\SetKwData{visited}{visited}\SetKwData{S}{S}\SetKwData{v}{v}\SetKwData{u}{u}\SetKwData{U}{U}\SetKwData{size}{size}\SetKwData{G}{G}\SetKwData{i}{i}\SetKwData{j}{j}\SetKwData{n}{n}
			\SetKwFunction{Push}{push}\SetKwFunction{Pop}{pop}\SetKwFunction{Max}{max}
			\SetKwInOut{Input}{Wejście}\SetKwInOut{Output}{Wyjście}\SetKwInput{KwData}{Zmienne pomocniczne}

			\Input{\G~- graf niezorientowany o~\n wierzchołkach}
			\Output{\maxSize~- rozmiar największej składowej spójnej}
			\KwData{\visited~- tablica logiczna odwiedzin wierzchołków
				    \\~~~~~~~~~~~~~~~~~~~~~~~~~~~~~~~~~~~~\S~- stos numerów wierzchołków
				    \\~~~~~~~~~~~~~~~~~~~~~~~~~~~~~~~~~~~~\i,~\j,~\u~- numery wierzchołków w~grafie
				    \\~~~~~~~~~~~~~~~~~~~~~~~~~~~~~~~~~~~~\size~- rozmiar badanej składowej}
			\BlankLine

			Utwórz tablicę \visited o~\n elementach\;
			Tablicę \visited wypełnij wartościami \textbf{false}\;
			Utwórz pusty stos \S\;
			$\maxSize \gets 0$\;
			\BlankLine

			\For{$\i\leftarrow 0$ \KwTo $\n - 1$}{
				\If{$\visited[\i]$}{
					\textbf{pass}\;
				}
				\BlankLine

				$\size \gets 1$\;
				$\S.\Push{\i}$\;
				$\visited[\i] \gets \textbf{true}$\;
				\BlankLine

				\While{$\S \neq \emptyset$}{
					$\j \gets \S.\Pop{}$\;
					\BlankLine

					\ForEach{$\u \in \G[\j].\U$}{
						\If{$\visited[\u]$}{
							\textbf{pass}\;
						}
						\BlankLine

						$\size \gets \size + 1$\;
						$\S.\Push{\u}$\;
						$\visited[\i] \gets \textbf{true}$\;
					}
				}
				\BlankLine

				\If{$\size > \maxSize$}{
					$\maxSize \gets \size$\;
				}
			}
			\BlankLine

			\Return \maxSize\;
			\BlankLine

			\caption{Algorytm badania spójności sieci}
			\label{alg:spojnosc}
		\end{algorithm}

	\section{Projekt testów}
	\label{projekt:testy}

	W trakcie testów należy zbadać prawdopodobieństwo spójności sieci oraz rozmiar największej składowej spójnej, w~zależności od ilości wierzchołków \texttt{n} oraz długości promienia \texttt{r}. W~związku z~tym, przeprowadzone zostaną dwie serie testów, podczas to których analizowany będzie wpływ każdego z~parametrów osobno.

	\subsection{Testy sieci o~zmieniającym się rozmiarze}
		\label{projekt:testy:rozmiar}

		\noindent Program testujący będzie przyjmował pięć zmiennych wejściowych:
		\begin{enumerate}
			\item Początkową liczba wierzchołków - \texttt{begin\_n}
			\item Końcową liczba wierzchołków - \texttt{end\_n}
			\item Krok zwiększania liczby wierzchołków - \texttt{step\_n}
			\item Promień - \texttt{r}
			\item Liczbę testów - \texttt{m}
		\end{enumerate}

		Dla zmieniającego się rozmiaru sieci (od \texttt{begin\_n} do \texttt{end\_n}, z~krokiem \texttt{step\_n}) i~stałego promienia \texttt{r} generowane będzie tyle grafów, ile wynosi liczba testów \texttt{m}. Następnie badana będzie spójność wygenerowanych grafów oraz rozmiar największej składowej spójnej. Otrzymane wyniki zostaną potem uśrednione, dając w~rezultacie listę prawdopodobieństw spójności grafu oraz średnich rozmiarów maksymalnych składowych spójnych. Każdy element tej listy będzie odpowiadać przypadkowi grafów o~innej liczbie wierzchołków.

	\subsection{Testy sieci o~zmieniającym się promieniu}
		\label{projekt:testy:promien}

		\noindent Program testujący będzie przyjmował pięć zmiennych wejściowych:
		\begin{enumerate}
			\item Początkową długość promienia - \texttt{begin\_r}
			\item Końcową długość promienia - \texttt{end\_r}
			\item Krok zwiększania długości promienia - \texttt{step\_r}
			\item Liczbę wierzchołków - \texttt{n}
			\item Liczbę testów - \texttt{m}
		\end{enumerate}

		Dla stałej liczby wierzchołków \texttt{n} oraz zmieniającej się długości promienia (od \texttt{begin\_r} do \texttt{end\_r}, z~krokiem \texttt{step\_r}) generowane będzie tyle grafów, ile wynosi liczba testów \texttt{m}. Następnie badana będzie spójność wygenerowanych grafów oraz rozmiar największej składowej spójnej. Otrzymane wyniki zostaną potem uśrednione, dając w~rezultacie listę prawdopodobieństw spójności grafu oraz średnich rozmiarów maksymalnych składowych spójnych. Każdy element tej listy będzie odpowiadać przypadkowi grafów o~innej długości promienia.

	\subsection{Prezentacja wyników}
		\label{projekt:testy:wyniki}

		Dane zebrane po wykonywaniu testów opisanych w~punktach~\ref{projekt:testy:rozmiar} i~\ref{projekt:testy:promien} przedstawione zostaną na odpowiednich wykresach. Planuje się wykonanie sześć wykresów, każdy opisujący inną zależność:
		\begin{itemize}
			\item Prawdopodobieństwo spójności grafu w~zależności od zmieniającej się liczby wierzchołków oraz stałego promienia,
			\item Prawdopodobieństwo spójności grafu w~zależności od zmieniającego się promienia oraz stałej liczby wierzchołków,
			\item Prawdopodobieństwo spójności grafu w~zależności od zmieniającej się liczby wierzchołków i~zmieniającego się promienia,
			\item Rozmiar największej składowej spójnej w~zależności od zmieniającej się liczby wierzchołków oraz stałego promienia,
			\item Rozmiar największej składowej spójnej w~zależności od zmieniającego się promienia oraz stałej liczby wierzchołków,
			\item Rozmiar największej składowej spójnej w~zależności od zmieniającej się liczby wierzchołków oraz zmieniającego się promienia,
		\end{itemize}

		Wykresy będą przedstawiały, jaki wpływ na spójność grafu mają długość promienia oraz liczba wierzchołków. Przykładowo mając zadany promień oraz zmieniającą się liczbę wierzchołków grafu spodziewamy się, że prawdopodobieństwo spójności sieci będzie wzrastało wraz ze zwiększaniem liczby wierzchołków. Testy będą miały na celu poparcie naszych przypuszczeń.

		Omawiane badania wykonywane będą dla wystarczająco dużych zbiorów danych. Wykresy z~wynikami zapisywane będą w~plikach z~rozszerzeniem \texttt{.png}.


%!TEX root = ./report.tex

\section{Zmiany w projekcie}
	\label{final:zmiany}

	Jedną z istotnych zmian jakie zostały wprowadzone do projektu to zmiana struktury grafu. Na początku zakładaliśmy, że każdy wierzchołek będzie składał się z 2 współrzędnych oraz będzie zawierał listę sąsiadujących z nim wierzchołków. Rozwiązanie to było wystarczające, jednak w celu poprawienia szybkości obliczeń zastosowana została macierz sąsiedztwa(lista tablic). Podczas generowania grafu macierz sąsiedztwa była wypełniana wartościami logicznymi informującymi o tym, czy dany wierzchołek jest sąsiadem innego.\\
	Aktualna struktura grafu składa się z:
	\begin{itemize}
		\item Lista wierzchołków
		\item Lista tablic - jeśli występuje krawędź między wierzchołkiem o indeksie 'i' oraz 'j' to odpowiadająca komórka w tablicy jest oznaczana jako True
	\end{itemize}
	
	Złożoność pamięciowa takiego rozwiązania wynosi aż O(n*n). Natomiast zapisana w takiej formie jest łatwo serializowalna oraz kompresowalna, co pozwoliło na oszczędzenie miejsca w stosunku do zastosowanej wcześniej listy sąsiedztwa.\\
	Zmiana struktury grafu spowodowała, że konieczna była zmiana w algorytmie wyszukiwania składowej spójnej.

	% Napisz o tym, że zmiana struktury grafu pociągnęła zmiany w algorytmie wyszukiwania składowej stałej i napisz jaka to zmiana była

%!TEX root = ./report.tex

\chapter{Polecenia}

	\section{Polecenia}
		\label{polecenia:analizator}

		\noindent W celu uruchomienia programu należy uruchomić plik analyze\_graphs.py. Przyjmuje on następujące parametry:
		\begin{itemize}
			\item --start\_size - WYMAGANY. Liczba wierzchołków pierwszego generowanego grafu

			\item --stop\_size - WYMAGANY. Liczba wierzchołków ostatniego generowanego grafu

			\item --size\_step - OPCJONALNY. Liczba wierzchołków dodawana do kolejnego generowanego grafu.
			\item --start\_radius - WYMAGANY. Promień pierwszego generowanego grafu

			\item --stop\_radius - WYMAGANY. Promień ostatniego generowanego grafu

			\item --radius\_step - OPCJONALNY. Promień dodawany do kolejnego generowanego grafu

			\item --repeats - OPCJONALNY. Liczba testów wykonywanych w ramach jednego grafu

			\item --jobs - OPCJONALNY. Liczba wątków, na których będą wykonywane obliczenia

			\item --output\_dir - OPCJONALNY. Folder, do którego będą zapisywane wyniki

			\item --verbose - OPCJONALNY. Tryb wypisujący szczegółowe informacje w konsoli podczas analizowania grafów
		\end{itemize}

		Przykładowe wywołanie polecenia:\\
		python3 ./analyze\_graphs.py -v --start\_size=10000 --stop\_size=20000 --size\_step=10 --start\_radius=0.01 --stop\_radius=0.02 --radius\_step=0.001 --jobs=8 --repeats=100 --output\_dir=output\\
		Takie wywołanie spowoduje, że zostaną wowołane testy na grafach o rozmiarze od 10000 do 20000 wierzchołków z krokiem 1, promieniem wzrastającym o 0.001 od 0.01 do 0.02. Liczba testów wykonanych na jednym grafie będzie wynosić 100. Obliczenia będą wykonywane z użyciem 8 wątków. Wyniki zostaną zapisane w folderze output.

		


%!TEX root = ./report.tex

\chapter{Przebieg testowania}

	\section{Generowanie grafów}
		\label{testowanie:generacja}
		
		Przed właściwym testowaniem właściwości grafów należało przygotować dane, które będą analizowane. W tym celu został zaimplementowany generator grafów. Generowane grafy zapisywane są do pliku w celu ich późniejszego wykorzystania. Zostało to tak zaimplementowane ze względu na bardzo długi czas generacji grafów. Był to etap, który zajął najwięcej czasu w procesie testowania.\\
\\
TABELKA
\\
\\
Następnie po wygenerowaniu grafów następuje proces badania ich właściwości. W tym etapie w przetwarzanym grafie obliczany jest rozmiar największej składowej spójnej. Rozmiar ten również jest zapisywany do pliku. Mając w ten sposób przygotowane dane o analizowanych grafach można przedstawić wyniki w postaci wykresów obrazujących prawdopodobieństwo spójności sieci w zależności od liczby wierzchołków oraz promienia.
	\section{Przykład1}
		przykład od ilu do ilu wierzchołków, wykresy itp. itd

	\section{Przykład2}

	\section{Wnioski}
		Proces generowania grafów oraz ich testowania zajął kilka dni. W celu ich automatyzacji zostały zaimplementowane polecenia pozwalające na generowanie grafów ze zmieniającym się ich rozmiarem oraz promieniem. Dzięki temu można było pozostawić obliczenia komputerowi na kilka godzin. Jednak zdarzały się problemy podczas testowania, które utrudniały pracę. Między innymi zdarzały się problemy z wyciekiem pamięci, przez co konieczne było ponowne wykonanywanie obliczeń.\\
	\\
	Wnioski na temat prawdopodobiensta spojnosci sieci itp itd.


\printbibliography[heading=bibintoc]

\end{document}
