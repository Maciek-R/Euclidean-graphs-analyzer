%!TEX encoding = UTF-8~Unicode
%!TEX program = pdflatex
%!BIB program = biber

\documentclass[11pt,a4paper,twoside]{report}
\usepackage[a4paper, top=25mm, bottom=25mm, inner=30mm, outer=30mm, twoside]{geometry}

\usepackage[polish]{babel}
\usepackage[T1]{fontenc}
\usepackage[utf8]{inputenc}
\usepackage{lmodern}

\usepackage{amsmath}
\usepackage{url}
\usepackage{fancyhdr}
\usepackage{footnote}
\usepackage{float}
% Kolory kodu
\usepackage{xcolor}
% Linki w~bibliografii
\usepackage{url}
% Przecinki zamiast kropek w~liczbach
\usepackage{icomma}
% Generalna typografia (automagia)
\usepackage{microtype}
\usepackage[unicode,pdfusetitle,bookmarksopen,bookmarksnumbered]{hyperref}
\usepackage{graphicx}
\usepackage{array}
\usepackage[font=footnotesize,format=hang,labelfont=bf]{caption}
% Wcięcie pierwszego akapitu każdej sekcji
\usepackage{indentfirst}
\newcommand{\HRule}[1]{\rule{\linewidth}{#1}}

\fancyhf{}
\renewcommand{\headrulewidth}{0pt}
\pagestyle{fancy}
\fancyfoot[LE,RO]{\thepage}

%%% Bibliografia
% Zamiana cudzysłowów (dla biblatexa)
\usepackage{csquotes}
% Lepsza bibliografia
\usepackage[backend=biber,style=numeric,sorting=nty,abbreviate=false,language=polish]{biblatex}
\addbibresource{report.bib}
\renewcommand*{\newunitpunct}{\addcomma\space}
\DeclareQuoteAlias{croatian}{polish}

\begin{document}
%-------------------------------------------------------------------------------
\title{ \normalsize \textsc{Grafy i Sieci\\Dokumentacja projektowa}
		\\ [2.0cm]
		\LARGE \textbf{Badanie właściwości grafów euklidesowych}
		\\ [2.0cm]
		\normalsize \today \vspace*{3\baselineskip}}
\date{}
\author{Adam Kowalewski\\Maciej Ruszczyk}
\maketitle
\tableofcontents
\newpage
%-------------------------------------------------------------------------------

\chapter{Wprowadzenie}

Celem zadania projektowego było zaimplementowanie generatora sieci euklidesowych, a~następnie zbadanie prawdopodobieństwa spójności sieci i~określenie rozmiaru największej składowej spójnej w~zależności od liczby i~zasięgu wierzchołków. Zarówno generator, jak i~oprogramowanie weryfikujące zostaną napisane w~języku Python. Głównym źródłem wiedzy będzie książka autorstwa Jacka Wojciechowskiego i~Krzysztofa Pieńkosza ,,Grafy i~Sieci''.\cite{WojciechowskiJacek2013Gis}

\printbibliography[heading=bibintoc]

\end{document}
