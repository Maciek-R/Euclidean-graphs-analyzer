\section{Testy rozwiązania}
	\label{final:testy}

	\subsection{Generowanie grafów}
		\label{final:testy:generowanie}

		% Przed właściwym testowaniem właściwości grafów należało przygotować dane, które będą analizowane. W tym celu został zaimplementowany generator grafów. Generowane grafy zapisywane są do pliku w celu ich późniejszego wykorzystania. Zostało to tak zaimplementowane ze względu na bardzo długi czas generacji grafów. Był to etap, który zajął najwięcej czasu w procesie testowania.\\
% \\
% TABELKA
% \\
% \\
% Następnie po wygenerowaniu grafów następuje proces badania ich właściwości. W tym etapie w przetwarzanym grafie obliczany jest rozmiar największej składowej spójnej. Rozmiar ten również jest zapisywany do pliku. Mając w ten sposób przygotowane dane o analizowanych grafach można przedstawić wyniki w postaci wykresów obrazujących prawdopodobieństwo spójności sieci w zależności od liczby wierzchołków oraz promienia.
	\subsection{Przykład1}
		\label{final:testy:przyklad1}
		% przykład od ilu do ilu wierzchołków, wykresy itp. itd

	\subsection{Przykład2}
		\label{final:testy:przyklad2}

	\subsection{Wnioski}
		\label{final:testy:wnioski}

		% Proces generowania grafów oraz ich testowania zajął kilka dni. W celu ich automatyzacji zostały zaimplementowane polecenia pozwalające na generowanie grafów ze zmieniającym się ich rozmiarem oraz promieniem. Dzięki temu można było pozostawić obliczenia komputerowi na kilka godzin. Jednak zdarzały się problemy podczas testowania, które utrudniały pracę. Między innymi zdarzały się problemy z wyciekiem pamięci, przez co konieczne było ponowne wykonanywanie obliczeń.\\
	% \\
	% Wnioski na temat prawdopodobiensta spojnosci sieci itp itd.
