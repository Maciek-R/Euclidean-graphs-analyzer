\section{Testy rozwiązania}
	\label{final:testy}

	\subsection{Generowanie grafów}
		\label{final:testy:generowanie}

		Istotnym etapem projektu było przygotowywanie danych testowych. Do tego celu został zaimplementowany generator grafów(klasa GraphGenerator). Generowane grafy były zapisywane do pliku w celu ich późniejszego wykorzystania do analizy. Etap generowania grafów był etapem, który zajął najwięcej czasu w procesie testowania i wynosił on kilka dni. Grafy, które zostały wygenerowane zajęły około 10 GB pamięci na dysku.
% \\
% TABELKA
% \\
% \\
	\subsection{Przykład1}
		\label{final:testy:przyklad1}
		% przykład od ilu do ilu wierzchołków, wykresy itp. itd

	\subsection{Przykład2}
		\label{final:testy:przyklad2}

	\subsection{Wnioski}
		\label{final:testy:wnioski}
	% \\
	% Wnioski na temat prawdopodobiensta spojnosci sieci itp itd.
