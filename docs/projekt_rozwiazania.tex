%!TEX root = ./report.tex

\chapter{Projekt rozwiązania}

	\section{Struktury danych}
		\label{projekt:struktury}

		%TODO: Dać rysunek poglądowy struktur danych

		\noindent W projekcie zostaną wykorzystane następujące struktury danych:
		\begin{itemize}
			\item Struktura opisująca graf - \texttt{G}. Składać będzie się ona z:
				\begin{itemize}
					\item Tablicy wierzchołków \texttt{V} o~rozmiarze \texttt{n},
 				\end{itemize}

			\item Struktura opisująca wierzchołek - \texttt{v}. Składać będzie się ona z:
				\begin{itemize}
					\item Współrzędnej $\texttt{x} \in [0; 1]$
					\item Współrzędnej $\texttt{y} \in [0; 1]$
					\item Listy z~numerami sąsiadujących wierzchołków \texttt{U}
				\end{itemize}
		\end{itemize}

	\section{Opis algorytmów}

		Programy zaimplementowane w~ramach projektu wykonywać będą zasadniczo dwie czynności: generować sieć euklidesową oraz badać parametry spójności. Poniżej opisano zaproponowane algorytmy.

		\subsection{Generacja sieci euklidesowych}
			\label{projekt:algorytmy:generacja}

			%TODO: Dać listing algorytmu generacji
			%TODO: Dać przykładową sieć

			Algorytm ten będzie przyjmował dwa dodatnie parametry wejściowe: liczbę wierzchołków \texttt{n} oraz promień \texttt{r}. Na początku algorytm generuje odpowiednią ilość wierzchołków, w~sposób losowy, na obszarze 2D o~współrzędnych $(\texttt{x}, \texttt{y})$ mieszczących się w~zakresie $[0; 1]$. Następnie, dla każdej pary wierzchołków $(\texttt{v}, \texttt{u})$, obliczana jest odległość między nimi, w~oparciu o~metrykę euklidesową. Jeśli odległość ta jest mniejsza lub równa promieniowi, między wierzchołkami tworzona jest krawędź (dodawany jest nowy sąsiad).

		\subsection{Badanie spójności sieci}
			\label{projekt:algorytmy:spojnosc}

			Algorytm ten zostanie zaimplementowany w~oparciu o~technikę badania grafu w głąb\cite[s.~270]{WojciechowskiJacek2013Gis}. Na wejściu przyjmować będzie sieć wygenerowaną w~punkcie~\ref{projekt:algorytmy:generacja}. Na wyjściu zaś zwrócony zostanie rozmiar największej składowej spójnej. Jeśli dany graf jest spójny, to otrzymana wartość będzie równa ilości krawędzi. Pseudokod opisujący algorytm przedstawiono na listingu~\ref{alg:spojnosc}.

			\begin{algorithm}
				\SetKwData{maxSize}{maxSize}\SetKwData{visited}{visited}\SetKwData{S}{S}\SetKwData{v}{v}\SetKwData{u}{u}\SetKwData{U}{U}\SetKwData{size}{size}\SetKwData{G}{G}\SetKwData{i}{i}\SetKwData{j}{j}\SetKwData{n}{n}
				\SetKwFunction{Push}{push}\SetKwFunction{Pop}{pop}\SetKwFunction{Max}{max}
				\SetKwInOut{Input}{Wejście}\SetKwInOut{Output}{Wyjście}\SetKwInput{KwData}{Zmienne pomocniczne}

				\Input{\G~- graf niezorientowany o~\n wierzchołkach}
				\Output{\maxSize~- rozmiar największej składowej spójnej}
				\KwData{\visited~- tablica logiczna odwiedzin wierzchołków
					    \\~~~~~~~~~~~~~~~~~~~~~~~~~~~~~~~~~~~~\S~- stos numerów wierzchołków
					    \\~~~~~~~~~~~~~~~~~~~~~~~~~~~~~~~~~~~~\i,~\j,~\u~- numery wierzchołków w~grafie
					    \\~~~~~~~~~~~~~~~~~~~~~~~~~~~~~~~~~~~~\size~- rozmiar badanej składowej}
				\BlankLine

				Utwórz tablicę \visited o~\n elementach\;
				Tablicę \visited wypełnij wartościami \textbf{false}\;
				Utwórz pusty stos \S\;
				$\maxSize \gets 0$\;
				\BlankLine

				\For{$\i\leftarrow 0$ \KwTo $\n - 1$}{
					\If{$\visited[\i]$}{
						\textbf{pass}\;
					}
					\BlankLine

					$\size \gets 1$\;
					$\S.\Push{\i}$\;
					$\visited[\i] \gets \textbf{true}$\;
					\BlankLine

					\While{$\S \neq \emptyset$}{
						$\j \gets \S.\Pop{}$\;
						\BlankLine

						\ForEach{$\u \in \G[\j].\U$}{
							\If{$\visited[\u]$}{
								\textbf{pass}\;
							}
							\BlankLine

							$\size \gets \size + 1$\;
							$\S.\Push{\u}$\;
							$\visited[\i] \gets \textbf{true}$\;
						}
					}
					\BlankLine

					\If{$\size > \maxSize$}{
						$\maxSize \gets \size$\;
					}
				}
				\BlankLine

				\Return \maxSize\;
				\BlankLine

				\caption{Algorytm badania spójności sieci}
				\label{alg:spojnosc}
			\end{algorithm}

	\section{Projekt testów}
		\label{projekt:testy}

		W trakcie testów należy zbadać prawdopodobieństwo spójności sieci oraz rozmiar największej składowej spójnej, w~zależności od ilości wierzchołków \texttt{n} oraz długości promienia \texttt{r}. W~związku z~tym, przeprowadzone zostaną dwie serie testów, podczas to których analizowany będzie wpływ każdego z~parametrów osobno.

		\subsection{Testy sieci o~zmieniającym się rozmiarze}
			\label{projekt:testy:rozmiar}

			\noindent Program testujący będzie przyjmował pięć zmiennych wejściowych:
			\begin{enumerate}
				\item Początkową liczba wierzchołków - \texttt{begin\_n}
				\item Końcową liczba wierzchołków - \texttt{end\_n}
				\item Krok zwiększania liczby wierzchołków - \texttt{step\_n}
				\item Promień - \texttt{r}
				\item Liczbę testów - \texttt{m}
			\end{enumerate}

			Dla zmieniającego się rozmiaru sieci (od \texttt{begin\_n} do \texttt{end\_n}, z~krokiem \texttt{step\_n}) i~stałego promienia \texttt{r} generowane będzie tyle grafów, ile wynosi liczba testów \texttt{m}. Następnie badana będzie spójność wygenerowanych grafów oraz rozmiar największej składowej spójnej. Otrzymane wyniki zostaną potem uśrednione, dając w~rezultacie listę prawdopodobieństw spójności grafu oraz średnich rozmiarów maksymalnych składowych spójnych. Każdy element tej listy będzie odpowiadać przypadkowi grafów o~innej liczbie wierzchołków.

		\subsection{Testy sieci o~zmieniającym się promieniu}
			\label{projekt:testy:promien}

			\noindent Program testujący będzie przyjmował pięć zmiennych wejściowych:
			\begin{enumerate}
				\item Początkową długość promienia - \texttt{begin\_r}
				\item Końcową długość promienia - \texttt{end\_r}
				\item Krok zwiększania długości promienia - \texttt{step\_r}
				\item Liczbę wierzchołków - \texttt{n}
				\item Liczbę testów - \texttt{m}
			\end{enumerate}

			Dla stałej liczby wierzchołków \texttt{n} oraz zmieniającej się długości promienia (od \texttt{begin\_r} do \texttt{end\_r}, z~krokiem \texttt{step\_r}) generowane będzie tyle grafów, ile wynosi liczba testów \texttt{m}. Następnie badana będzie spójność wygenerowanych grafów oraz rozmiar największej składowej spójnej. Otrzymane wyniki zostaną potem uśrednione, dając w~rezultacie listę prawdopodobieństw spójności grafu oraz średnich rozmiarów maksymalnych składowych spójnych. Każdy element tej listy będzie odpowiadać przypadkowi grafów o~innej długości promienia.

		\subsection{Prezentacja wyników}
			\label{projekt:testy:wyniki}

			Dane zebrane po wykonywaniu testów opisanych w~punktach~\ref{projekt:testy:rozmiar} i~\ref{projekt:testy:promien} przedstawione zostaną na odpowiednich wykresach. Planuje się wykonanie sześć wykresów, każdy opisujący inną zależność:
			\begin{itemize}
				\item Prawdopodobieństwo spójności grafu w~zależności od zmieniającej się liczby wierzchołków oraz stałego promienia,
				\item Prawdopodobieństwo spójności grafu w~zależności od zmieniającego się promienia oraz stałej liczby wierzchołków,
				\item Prawdopodobieństwo spójności grafu w~zależności od zmieniającej się liczby wierzchołków i~zmieniającego się promienia,
				\item Rozmiar największej składowej spójnej w~zależności od zmieniającej się liczby wierzchołków oraz stałego promienia,
				\item Rozmiar największej składowej spójnej w~zależności od zmieniającego się promienia oraz stałej liczby wierzchołków,
				\item Rozmiar największej składowej spójnej w~zależności od zmieniającej się liczby wierzchołków oraz zmieniającego się promienia,
			\end{itemize}

			Wykresy będą przedstawiały, jaki wpływ na spójność grafu mają długość promienia oraz liczba wierzchołków. Przykładowo mając zadany promień oraz zmieniającą się liczbę wierzchołków grafu spodziewamy się, że prawdopodobieństwo spójności sieci będzie wzrastało wraz ze zwiększaniem liczby wierzchołków. Testy będą miały na celu poparcie naszych przypuszczeń.

			Omawiane badania wykonywane będą dla wystarczająco dużych zbiorów danych. Wykresy z~wynikami zapisywane będą w~plikach z~rozszerzeniem \texttt{.png}.
