%!TEX root = ./report.tex

\chapter{Podsumowanie}
	\label{podsumowanie}

	Wykonanie projektu wymagało od studentów zarówno wiedzy potrzebnej do zrozumienia dziedziny problemu, a następnie wykorzystania jej w celu implementacji algorytmu pozwalającego na przetestowanie własności sieci euklidesowych.\\
	W pierwszym etapie pracy należało skupić się na zagadnieniu sieci euklidesowych tj. co to jest, jak ją wygenerować, jakie posiada własności.\\
	Po zrozumieniu dziedziny problemu można było skupić się na implementacji generacji grafów, obliczenia spójności grafu oraz maksymalnego rozmiaru składowej spójnej.\\
	W kolejnym etapie skupiono się procesie testowania grafów w dużej ilości oraz zawierających znaczną liczbę wierzchołków w celu sprawdzenia jak zachowują się grafy z określoną liczbą wierzchołków oraz określonym promieniem i jak wpływa zmiana liczby wierzchołków i promienia na spójność grafu. W tym etapie również zaimplementowano mechanizm tworzenia wykresów w celu zobrazowania własności grafów.\\
	Kolejne etapy skupiały się na wzroście wydajności czasu generowania grafów oraz ich analizowania. Zastosowano tu między innymi mechanizmy wielowątkowości oraz zapisu grafów na dysku.\\
	Cały projekt wymagał nie tylko umiejętności związanych z analizą właściwości grafów, ale również umiejętności programistycznych.
