\section{Projekt testów}
	\label{projekt:testy}

	W trakcie testów należy zbadać prawdopodobieństwo spójności sieci oraz rozmiar największej składowej spójnej, w~zależności od ilości wierzchołków \texttt{n} oraz długości promienia \texttt{r}. W~związku z~tym, przeprowadzone zostaną dwie serie testów, podczas to których analizowany będzie wpływ każdego z~parametrów osobno.

	\subsection{Testy sieci o~zmieniającym się rozmiarze}
		\label{projekt:testy:rozmiar}

		\noindent Program testujący będzie przyjmował pięć zmiennych wejściowych:
		\begin{enumerate}
			\item Początkową liczba wierzchołków - \texttt{begin\_n}
			\item Końcową liczba wierzchołków - \texttt{end\_n}
			\item Krok zwiększania liczby wierzchołków - \texttt{step\_n}
			\item Promień - \texttt{r}
			\item Liczbę testów - \texttt{m}
		\end{enumerate}

		Dla zmieniającego się rozmiaru sieci (od \texttt{begin\_n} do \texttt{end\_n}, z~krokiem \texttt{step\_n}) i~stałego promienia \texttt{r} generowane będzie tyle grafów, ile wynosi liczba testów \texttt{m}. Następnie badana będzie spójność wygenerowanych grafów oraz rozmiar największej składowej spójnej. Otrzymane wyniki zostaną potem uśrednione, dając w~rezultacie listę prawdopodobieństw spójności grafu oraz średnich rozmiarów maksymalnych składowych spójnych. Każdy element tej listy będzie odpowiadać przypadkowi grafów o~innej liczbie wierzchołków.

	\subsection{Testy sieci o~zmieniającym się promieniu}
		\label{projekt:testy:promien}

		\noindent Program testujący będzie przyjmował pięć zmiennych wejściowych:
		\begin{enumerate}
			\item Początkową długość promienia - \texttt{begin\_r}
			\item Końcową długość promienia - \texttt{end\_r}
			\item Krok zwiększania długości promienia - \texttt{step\_r}
			\item Liczbę wierzchołków - \texttt{n}
			\item Liczbę testów - \texttt{m}
		\end{enumerate}

		Dla stałej liczby wierzchołków \texttt{n} oraz zmieniającej się długości promienia (od \texttt{begin\_r} do \texttt{end\_r}, z~krokiem \texttt{step\_r}) generowane będzie tyle grafów, ile wynosi liczba testów \texttt{m}. Następnie badana będzie spójność wygenerowanych grafów oraz rozmiar największej składowej spójnej. Otrzymane wyniki zostaną potem uśrednione, dając w~rezultacie listę prawdopodobieństw spójności grafu oraz średnich rozmiarów maksymalnych składowych spójnych. Każdy element tej listy będzie odpowiadać przypadkowi grafów o~innej długości promienia.

	\subsection{Prezentacja wyników}
		\label{projekt:testy:wyniki}

		Dane zebrane po wykonywaniu testów opisanych w~punktach~\ref{projekt:testy:rozmiar} i~\ref{projekt:testy:promien} przedstawione zostaną na odpowiednich wykresach. Planuje się wykonanie sześć wykresów, każdy opisujący inną zależność:
		\begin{itemize}
			\item Prawdopodobieństwo spójności grafu w~zależności od zmieniającej się liczby wierzchołków oraz stałego promienia,
			\item Prawdopodobieństwo spójności grafu w~zależności od zmieniającego się promienia oraz stałej liczby wierzchołków,
			\item Prawdopodobieństwo spójności grafu w~zależności od zmieniającej się liczby wierzchołków i~zmieniającego się promienia,
			\item Rozmiar największej składowej spójnej w~zależności od zmieniającej się liczby wierzchołków oraz stałego promienia,
			\item Rozmiar największej składowej spójnej w~zależności od zmieniającego się promienia oraz stałej liczby wierzchołków,
			\item Rozmiar największej składowej spójnej w~zależności od zmieniającej się liczby wierzchołków oraz zmieniającego się promienia,
		\end{itemize}

		Wykresy będą przedstawiały, jaki wpływ na spójność grafu mają długość promienia oraz liczba wierzchołków. Przykładowo mając zadany promień oraz zmieniającą się liczbę wierzchołków grafu spodziewamy się, że prawdopodobieństwo spójności sieci będzie wzrastało wraz ze zwiększaniem liczby wierzchołków. Testy będą miały na celu poparcie naszych przypuszczeń.

		Omawiane badania wykonywane będą dla wystarczająco dużych zbiorów danych. Wykresy z~wynikami zapisywane będą w~plikach z~rozszerzeniem \texttt{.png}.
