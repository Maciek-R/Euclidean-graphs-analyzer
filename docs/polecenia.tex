%!TEX root = ./report.tex

\chapter{Polecenia}

	\section{Polecenia}
		\label{polecenia:analizator}

		\noindent W celu uruchomienia programu należy uruchomić plik analyze\_graphs.py. Przyjmuje on następujące parametry:
		\begin{itemize}
			\item --start\_size - WYMAGANY. Liczba wierzchołków pierwszego generowanego grafu

			\item --stop\_size - WYMAGANY. Liczba wierzchołków ostatniego generowanego grafu

			\item --size\_step - OPCJONALNY. Liczba wierzchołków dodawana do kolejnego generowanego grafu.
			\item --start\_radius - WYMAGANY. Promień pierwszego generowanego grafu

			\item --stop\_radius - WYMAGANY. Promień ostatniego generowanego grafu

			\item --radius\_step - OPCJONALNY. Promień dodawany do kolejnego generowanego grafu

			\item --repeats - OPCJONALNY. Liczba testów wykonywanych w ramach jednego grafu

			\item --jobs - OPCJONALNY. Liczba wątków, na których będą wykonywane obliczenia

			\item --output\_dir - OPCJONALNY. Folder, do którego będą zapisywane wyniki

			\item --verbose - OPCJONALNY. Tryb wypisujący szczegółowe informacje w konsoli podczas analizowania grafów
		\end{itemize}

		Przykładowe wywołanie polecenia:\\
		python3 ./analyze\_graphs.py -v --start\_size=10000 --stop\_size=20000 --size\_step=10 --start\_radius=0.01 --stop\_radius=0.02 --radius\_step=0.001 --jobs=8 --repeats=100 --output\_dir=output\\
		Takie wywołanie spowoduje, że zostaną wowołane testy na grafach o rozmiarze od 10000 do 20000 wierzchołków z krokiem 1, promieniem wzrastającym o 0.001 od 0.01 do 0.02. Liczba testów wykonanych na jednym grafie będzie wynosić 100. Obliczenia będą wykonywane z użyciem 8 wątków. Wyniki zostaną zapisane w folderze output.

		

