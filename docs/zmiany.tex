%!TEX root = ./report.tex

\section{Zmiany w projekcie}
	\label{final:zmiany}

	Jedną z istotnych zmian jakie zostały wprowadzone do projektu to zmiana struktury grafu. Na początku zakładaliśmy, że każdy wierzchołek będzie składał się z 2 współrzędnych oraz będzie zawierał listę sąsiadujących z nim wierzchołków. Rozwiązanie to było wystarczające, jednak w celu poprawienia szybkości obliczeń zastosowana została macierz sąsiedztwa(lista tablic). Podczas generowania grafu macierz sąsiedztwa była wypełniana wartościami logicznymi informującymi o tym, czy dany wierzchołek jest sąsiadem innego.\\
	Aktualna struktura grafu składa się z:
	\begin{itemize}
		\item Lista wierzchołków
		\item Lista tablic - jeśli występuje krawędź między wierzchołkiem o indeksie 'i' oraz 'j' to odpowiadająca komórka w tablicy jest oznaczana jako True
	\end{itemize}
	
	Złożoność pamięciowa takiego rozwiązania wynosi aż O(n*n). Natomiast zapisana w takiej formie jest łatwo serializowalna oraz kompresowalna, co pozwoliło na oszczędzenie miejsca w stosunku do zastosowanej wcześniej listy sąsiedztwa.\\
	Zmiana struktury grafu spowodowała, że konieczna była niewielka zmiana w algorytmie wyszukiwania składowej spójnej. Zamiast listy odwiedzonych wierzchołków w tablicy ustawiana jest wartość True jeśli dany wierzchołek jest sąsiadem innego. Zasada działania algorytmu DFS została taka sama.
