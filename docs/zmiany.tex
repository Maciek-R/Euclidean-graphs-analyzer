%!TEX root = ./report.tex

\section{Zmiany w projekcie}
	\label{final:zmiany}

	Jedną z istotnych zmian jakie zostały wprowadzone do projektu to zmiana struktury grafu. Na początku zakładaliśmy, że każdy wierzchołek będzie składał się z 2 współrzędnych oraz będzie zawierał listę sąsiadujących z nim wierzchołków. Rozwiązanie to było wystarczające, jednak w celu poprawienia szybkości obliczeń zastosowana została macierz sąsiedztwa. Podczas generowania grafu macierz sąsiedztwa była wypełniana wartościami logicznymi, informującymi o tym, czy dany wierzchołek jest sąsiadem innego.

	% W sumie to jest prawie "macierz", bo to jest lista tablic
	% Podsumuj strukturę grafu tak, jak to zrobiono w poprzednim rozdziale
	% Napisz o złożoności pamięciowej. Co prawda aktualne rozwiązanie zajmuje dużo miejsca w pamięci RAM (bo dla każdego wierzchołka, czy jest sąsiadem czy nie, mamy informację), ale jest znacznie łatwiej serializowalna, a przede wszystkim kompresowalna, więc finalnie na dysku twardym taki graf zajmuje mniej miejsca niż w przypadku listy sąsiedztwa.
	% Napisz o tym, że zmiana struktury grafu pociągnęła zmiany w algorytmie wyszukiwania składowej stałej i napisz jaka to zmiana była
