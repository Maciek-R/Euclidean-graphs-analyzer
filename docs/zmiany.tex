%!TEX root = ./report.tex

\section{Zmiany w projekcie}
	\label{final:zmiany}

	Jedną z istotnych zmian jakie zostały wprowadzone do projektu to zmiana struktury grafu. Na początku zakładaliśmy, że każdy wierzchołek będzie składał się z 2 współrzędnych oraz będzie zawierał listę sąsiadujących z nim wierzchołków. Rozwiązanie to było wystarczające, jednak nieefektywne z punktu widzenia naszych badań. W celu poprawienia szybkości obliczeń zastosowana została logiczna macierz sąsiedztwa. Ponadto udało się w dużej mierze zredukować zapotrzebowania pamięciowe dla każdego z grafów, zwłaszcza co do miejsca zajmowanego na dysku (o czym więcej napisano w~\ref{final:struktura:klasy}). Jest to niezgodne z intuicją (ponieważ od teraz mamy stałą złożoność pamięciową $O(\texttt{N}^2)$), jednak prosta macierz wartości logicznych okazuje się być znacznie łatwiej serializowalna i kompresowalna niż lista wierzchołków.

	Zmiana struktury grafu pociągnęła za sobą drobne zmiany w algorytmie wyszukiwania maksymalnej składowej spójnej. Wcześniej, w trakcie przechodzenia w głąb, kolejne wierzchołki pobierane były z listy sąsiadów badanego w danej iteracji. Teraz, jako że takowa lista nie istnieje, wierzchołki sąsiednie wyszukiwane są w wierszach macierzy sąsiedztwa.

	Podsumowując dokonane zmiany, struktura grafu będzie następująca:
	\begin{itemize}
		\item Lista wierzchołków \texttt{nodes} o rozmiarze \texttt{N}
		\item Macierz sąsiedztwa \texttt{edges} o rozmiarze \texttt{N x N} - jeśli występuje krawędź między wierzchołkiem o numerze \texttt{i} oraz \texttt{j}, to odpowiadająca komórka w tablicy \texttt{edges[i][j]} jest oznaczana jako logiczna prawda. Jako, że badane przez nas grafy będą nieskierowane, w takim przypadku również komórka \texttt{edges[j][i]} będzie zawierać logiczną prawdę.
	\end{itemize}

