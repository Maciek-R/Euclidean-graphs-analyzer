\section{Opis algorytmów}
	\label{projekt:algorytmy}

	Programy zaimplementowane w~ramach projektu wykonywać będą zasadniczo dwie czynności: generować sieć euklidesową oraz badać parametry spójności. Poniżej opisano zaproponowane algorytmy.

	\subsection{Generacja sieci euklidesowych}
		\label{projekt:algorytmy:generacja}

		%TODO: Dać listing algorytmu generacji
		%TODO: Dać przykładową sieć

		Algorytm ten będzie przyjmował dwa dodatnie parametry wejściowe: liczbę wierzchołków \texttt{n} oraz promień \texttt{r}. Na początku algorytm generuje odpowiednią ilość wierzchołków, w~sposób losowy, na obszarze 2D o~współrzędnych $(\texttt{x}, \texttt{y})$ mieszczących się w~zakresie $[0; 1]$. Następnie, dla każdej pary wierzchołków $(\texttt{v}, \texttt{u})$, obliczana jest odległość między nimi, w~oparciu o~metrykę euklidesową. Jeśli odległość ta jest mniejsza lub równa promieniowi, między wierzchołkami tworzona jest krawędź (dodawany jest nowy sąsiad).

	\subsection{Badanie spójności sieci}
		\label{projekt:algorytmy:spojnosc}

		Algorytm ten zostanie zaimplementowany w~oparciu o~technikę badania grafu w głąb\cite[s.~270]{WojciechowskiJacek2013Gis}. Na wejściu przyjmować będzie sieć wygenerowaną w~punkcie~\ref{projekt:algorytmy:generacja}. Na wyjściu zaś zwrócony zostanie rozmiar największej składowej spójnej. Jeśli dany graf jest spójny, to otrzymana wartość będzie równa ilości krawędzi. Pseudokod opisujący algorytm przedstawiono na listingu~\ref{alg:spojnosc}.

		\begin{algorithm}
			\SetKwData{maxSize}{maxSize}\SetKwData{visited}{visited}\SetKwData{S}{S}\SetKwData{v}{v}\SetKwData{u}{u}\SetKwData{U}{U}\SetKwData{size}{size}\SetKwData{G}{G}\SetKwData{i}{i}\SetKwData{j}{j}\SetKwData{n}{n}
			\SetKwFunction{Push}{push}\SetKwFunction{Pop}{pop}\SetKwFunction{Max}{max}
			\SetKwInOut{Input}{Wejście}\SetKwInOut{Output}{Wyjście}\SetKwInput{KwData}{Zmienne pomocniczne}

			\Input{\G~- graf niezorientowany o~\n wierzchołkach}
			\Output{\maxSize~- rozmiar największej składowej spójnej}
			\KwData{\visited~- tablica logiczna odwiedzin wierzchołków
				    \\~~~~~~~~~~~~~~~~~~~~~~~~~~~~~~~~~~~~\S~- stos numerów wierzchołków
				    \\~~~~~~~~~~~~~~~~~~~~~~~~~~~~~~~~~~~~\i,~\j,~\u~- numery wierzchołków w~grafie
				    \\~~~~~~~~~~~~~~~~~~~~~~~~~~~~~~~~~~~~\size~- rozmiar badanej składowej}
			\BlankLine

			Utwórz tablicę \visited o~\n elementach\;
			Tablicę \visited wypełnij wartościami \textbf{false}\;
			Utwórz pusty stos \S\;
			$\maxSize \gets 0$\;
			\BlankLine

			\For{$\i\leftarrow 0$ \KwTo $\n - 1$}{
				\If{$\visited[\i]$}{
					\textbf{pass}\;
				}
				\BlankLine

				$\size \gets 1$\;
				$\S.\Push{\i}$\;
				$\visited[\i] \gets \textbf{true}$\;
				\BlankLine

				\While{$\S \neq \emptyset$}{
					$\j \gets \S.\Pop{}$\;
					\BlankLine

					\ForEach{$\u \in \G[\j].\U$}{
						\If{$\visited[\u]$}{
							\textbf{pass}\;
						}
						\BlankLine

						$\size \gets \size + 1$\;
						$\S.\Push{\u}$\;
						$\visited[\i] \gets \textbf{true}$\;
					}
				}
				\BlankLine

				\If{$\size > \maxSize$}{
					$\maxSize \gets \size$\;
				}
			}
			\BlankLine

			\Return \maxSize\;
			\BlankLine

			\caption{Algorytm badania spójności sieci}
			\label{alg:spojnosc}
		\end{algorithm}
